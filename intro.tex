\section{Introduccion}
Sea $(\X, \A,\mu)$ un espacio de medida. Una medida se llama invariante bajo una transformaci\'on $S:\X\rightarrow\X$ si
$\mu(A)=\mu(S^{-1}(A))$ para $A\in\A$
Se puede observar que para una funci\'on $f:[0,1]\rightarrow[0,1]$ no existe un medida invariante absolutamente
continua, si la grafica de $f$ es muy plana. Por ejemplo: para la transformaci\'on $f(x)=rx (mod 1) $ con $|r|<1$ una medida invariante no existe.

En 1957 S. Ulam propuso el problema de la existencia de medidas invariantes absolutamente continuas
para funciones definidas por funciones suficientemente simples\footnote{Ejemplo: funciones lineales a trozos o poligonales}
donde el gr\'afico no corte la l\'inea $y=x$ con una pendiente de valor absoluto menor que 1. La respuesta  literal de
esta pregunta es negativa. Se puede ilustrar, que para la siguiente transformaci\'on

$$f(x)=\begin{cases}
		 1-2x     & 0\leq x\leq\frac{5}{12}\\	
         (2-2x)/7 & \frac{5}{12}< x\leq 1
        \end{cases}$$

No existe una medida invariante absolutamente continua. Note que esta transformaci\'on cruza la
l\'inea $y=x$ en el punto $x=\frac{1}{3}$ con pendiente $f'(x)=-2$.

Muchos resultados en relaci\'on de la existencia de medidas absolutamente contin\'uas para ciertas clases de
transformaciones del  intervalo unitario en si mismo han sido proporcionadas (A. R\'enyi, Parry, Krzyzewski/Szlenk).
Sin embargo, se puede se\~nalar que el mejor resultado en esta direcci\'on fue obtenido por A. Lasota y J. Yorke

\begin{teo} Sea $S:[0,1]\rightarrow[0,1]$ una funci\'on a trozos de clases $C^2$ que satisfaga la condici\'on
$$\inf_{x\in[0,1]}|\frac{d}{dx}f(x)|>1$$
Entonces existe una medida invariante absolutamente continua sobre $f$
\end{teo}

La metodolog\'ia usada por Lasota/Yorke en la demostraci\'on difiere  de los trabajos antes mencionados. En primer lugar se
uso el hecho de que operador de  Perron-Frobenius correspondiente a la transformaci\'on  tiene la propiedad de contracci\'on
en la variaci\'on de la funci\'on.  Luego prueba que cierta sucesi\'on es relativamente compacta, as\'i se cumplen en la
hip\'otesis del teorema de Mazur.

Esto garantiza, que el promedio de las orbitas convergen fuertemente a una funci\'on limite, por medio  al uso de teorema
de Kakutani-Yosida.  Repitiendo este proceso se consigue una familia de funciones que acotan a la variaci\'on del operador
de Perron-Frobenius. Para finalizar, usando el principio de selecci\'on de Helly se puede conseguir una sucesi\'on de funciones,
que convergen a una funci\'on de variaci\'on finita.

El objetivo del presente trabajo es aclarar lo detalles de la  presente demostraci\'on con lo cual se presentar\'a en el primer
cap\'itulo una introducci\'on intuitiva a los s\'istemas din\'amicos, caos y el operador de Perron-Frobenius. En el siguiente
cap\'itulo se analizar\'a en detalle el operador de Perron-Frobenius. En el tercer cap\'itulo,
se presenta la noci\'on de ergodicidad junto con otros niveles de irregularidades. El cuarto cap\'itulo est\'a dedicado a estudio
del los teoremas de Mazur, Kakutani-Yosida, asi tambi\'en  como del Principio de Helly.

Finalmente, se presenta el quinto cap\'itulo se probar\'a el teorema de Lasota-Yorke junto con un contra ejemplo. Los ap\'endices expuestos tratan
de unas nociones b\'asicas de s\'stemas din\'amicos y de teor\'ia de la medida necesaria para el desarrollo  de los temas tratados.

