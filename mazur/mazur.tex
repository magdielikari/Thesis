\section{Teorema de Mazur}

\begin{dfn}[Clausura,$\ov{E}$] Si $\ov{E}=\{x\in\X:\forall B(x,\epsilon)\cap E\neq\emptyset\}$
\end{dfn}

\begin{obs} Equivalentemente la clausura se puede definir mediante $\ov{E}=E\cup E'$ donde $E'$ es el conjunto de los puntos de acumulaci\'on de E. Tambi\'en, es la intersecci\'on de todos los conjuntos cerrados que contienen a E.
\end{obs}

\begin{dfn}[Totalmente acotado]  Si  $\forall \epsilon>0,\; \exists x_1,x_2,\ldots,x_n \in \X$ tal que
$\cup^n_{i=1}{B(x_i,\epsilon)}\supseteq\X$ siendo $(\X,d)$ un espacio m\'etrico.
\end{dfn}

\begin{obs} Se cumple que todo espacio totalmente acotado es tambi\'en acotado. Adem\'as, todo compacto es totalmente acotado.
Esta propiedad es \'util precisamente para demostrar compacidad, pues se tiene que existe equivalencia entre ser compacto
y ser totalmente acotado y completo.
\end{obs}

De hecho, para muchas demostraciones es precisamente \'esta caracterizaci\'on de compacidad la que se utiliza.

\begin{dfn}[Relativamente compacto] Si toda sucesi\'on de elementos de S tiene una subsucesi\'on que converge en X.
 Siendo $S$ un subconjunto de un espacio topol\'ogico $\X$
\end{dfn}

\begin{obs} Para espacios métricos tenemos definici\'on equivalente. A es relativamente compacto si y solo si su clausura es un compacto
\end{obs}

\begin{dfn}[Envolvente Convexa, co(A)] Es el conjunto de todas las combinaciones convexas de elementos de A, es decir, el conjunto de todas las sumas
$\sum^n_{i=1}{t_ix_i}$ donde $x_i\in A,\; t_i\geq 0,\; \sum{t_i}=1$ para un $n$ arbitrario.
\end{dfn}

\begin{teo}[Mazur] Sea $\X$ un espacio de banach con $A\subset\X$ es un compacto relativo, entonces $\ov{co}(A)$ es compacto.
\end{teo}\label{mazur}

\begin{proof} Por la definici\'on de espacio Banach, $\ov{co}(A)$ es completo asi que $co(A)\subset\X$. Por consiguiente, solo queda probar
que $co(A)$ es totalmente acotada.

Como $A$ es totalmente acotado entonces $\ov{A}$ es compacto. Adicionalmente, se escoje $\epsilon>0$, por lo tanto existe un subcojunto finito
$\{x_1,x_2,\ldots,x_n\}\subset A$ tal que
$$A\subset\cup^n_{i=1}{B(x_i,\frac{\epsilon}{4})}$$

Sea $B=co\{x_1,x_2,\ldots,x_n\}$ y note que (eso no lo entiendo)
$$\ov{co}(A)\subset\cup^n_{i=1}{B(co(A),\frac{\epsilon}{4})}$$

Sea $\upsilon:A\rightarrow\{1,2,\ldots,n\}$ tal que
$$g\in A\Rightarrow||g-x_{\upsilon(g)}||<\frac{\epsilon}{4}$$

Para $t\in co(A)$
$$t=\sum^m_{i=1}{a_it_i}$$
donde $t_i\in A$, $a_i\geq 0$ y $\sum^m_{i=1}{a_i}=1$

entonces
\begin{align*}
    ||t-\sum^m_{i=1}{a_ix_{\upsilon(t_i)}}||&=||\sum^m_{i=1}{a_i(t_i-x_{\upsilon(t_i))}}||\\
                                            &\leq \sum^m_{i=1}{|a_i| ||x_{\upsilon(t_i)}||}\\
                                            &\leq\frac{\epsilon}{4}
\end{align*}

Esto es para $t\in co(A)$
$$d(x,B)<\frac{\epsilon}{4}$$

En consecuencia, $co(A)\subset \cup^n_{i=1}{B(B,\frac{\epsilon}{4})}$ entonces
$\cup^n_{i=1}{B(co(A),\frac{\epsilon}{4})}\subset\cup^n_{i=1}{B(B,\frac{\epsilon}{4})}$
y por lo tanto $\ov{co}(A)\subset\cup^n_{i=1}{B(B,\frac{\epsilon}{2})}$

Ahora, la transformaci\'on
$$\mu:(a_1,a_2,\ldots,a_n,x_1,x_2,\ldots,x_n)\rightarrow\sum^{n}_{i=1}{a_ix_i}$$
es una transfomaci\'on continua de un conjunto compacto
$$[0,1]\times[0,1]\times\ldots\times[0,1]=\prod^n_{i=1}{x_i}\hbox{ sobre B}$$

En consecuencia, B es un compacto y por lo tanto es totalmente acotado. Entonces exiten $b_1,b_2\ldots b_n\in B$ tal que
$$B\subset \cup^n_{i=1}{B(b_i,\frac{\epsilon}{2})}$$.

Como $\ov{co}(A)\subset\cup^n_{i=1}{B(B,\frac{\epsilon}{2})}\subset\cup^n_{i=1}{B(b_i,\frac{\epsilon}{2})}$.
$\ov{co}(A)$ es totalmente acotado y por lo tanto es compacto

\end{proof}

 \newpage
