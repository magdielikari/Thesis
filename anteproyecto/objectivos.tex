\section{Objetivos}

\subsection{Objetivo General}
    Estudiar condiciones para la existencia de medidas invariantes de sistemas din�micos reales.
\subsection{Objetivos Espec�ficos}
    \begin{itemize}
        \item Estudiar algunos elementos del operador de Markov.
        \item Estudiar el operador de Frobenius-Perron.
        \item Demostrar la existencia de medidas invariantes absolutamente continuas.
        \item Experimentar en el computador algunos ejemplos de densidades invariantes para sistema din�micos reales.
    \end{itemize}

\section{Cronograma de Actividades}
    En esta secci�n, se expondr� las actividades ha ejecutar para la realizaci�n del Trabajo Especial de Grado, as� como, el cronograma de actividades para la elaboraci�n del mismo.
\subsection{Actividades a realizar}
    \begin{enumerate}
        \item Revisi�n de las notas ``Notas para un curso de teoria erg�dica'' de Fernando J. S�nchez S.
        \item Revisi�n del libro ``Probabilistic properties of deterministic systems''  de A. Lasota \& Michael C. Mackey
        \item Realizaci�n de los resultados del art�culo ``On the existence of invariant measures for piecewise monotonic transformations'',
         de A. Lasota y James A. Yorke.
        \item Realizaci�n de ejemplos sobre densidades invariantes para sistema din�micos reales.
        \item Redacci�n del Trabajo Especial de Grado.
    \end{enumerate}


\begin{table}
\begin{center}
        \begin{tabular}{|c|c|c|c|c|c|}
             \hline
             \backslashbox{Act}{Sem}     & 1 & 2 & 3 & 4 & 5 \\
             \hline
             \backslashbox{13-08}{17-08} & X &   &   &   & X \\
             \hline
             \backslashbox{20-08}{24-08} & X &   &   &   & X \\
             \hline
             \backslashbox{27-08}{31-08} &   & X &   &   & X \\
             \hline
             \backslashbox{03-09}{07-09} &   & X &   &   & X \\
             \hline
             \backslashbox{10-09}{14-09} &   &   & X &   & X \\
             \hline
             \backslashbox{17-09}{21-09} &   &   & X &   & X \\
             \hline
             \backslashbox{24-09}{28-09} &   &   &   & X & X \\
             \hline
             \backslashbox{01-10}{05-10} &   &   &   & X & X \\
             \hline
        \end{tabular}
\end{center}
\caption{Diagrama de Grantt}
\end{table}
