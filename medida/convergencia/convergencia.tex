\section{Convergencia de sucesiones de funciones}

\begin{dfn}[Convergencia Cesaro] La sucesi�n de funciones $\{f_n\}\rightarrow f$ si
    \begin{equation}
        \limi_{n\rightarrow\infty}{\frac{1}{n}\sum_{k=1}^{n}{<f_k,g>}}=<f,g> \qquad \forall g\in L^p
    \end{equation}
    donde $f_n, f\in L^p$ y $1\leq p<\infty$.
\end{dfn}

\begin{dfn}[Convergencia Debil]La sucesi�n de funciones $\{f_n\}\rightarrow f$ si
    \begin{equation}
        \limi_{n\rightarrow\infty}{<f_n,g>}=<f,g> \qquad \forall g\in L^p
    \end{equation}
    donde $f_n, f\in L^p$ y $1\leq p<\infty$.
\end{dfn}

\begin{dfn}[Convergencia Fuerte]La sucesi�n de funciones $\{f_n\}\rightarrow f$ si
    \begin{equation}
        \limi_{n\rightarrow\infty}{||f_k-f||_{L^p}}=0
    \end{equation}
    donde $f_n, f\in L^p$ y $1\leq p<\infty$.
\end{dfn}

\begin{obs} De la inecuaci�n de Cauchy-H�lder, se tiene
    $$|<f_n-f,g>|\leq ||f_n-f||_{L^p}\cdot||g||_{L^p}$$
    Y, asi, si $||f_n-f||_{L^p}$ converge a cero, entonces $<f_n-f,g>$ tambi�n. Entonces la convergencia fuerte implica convergencia d�bil.
    Usualmente la convergencia fuerte es m�s sencillo de probar. Sin embargo, la convergencia d�bil requiere
    la demonstraci�n que asegure que para toda $g\in L^p$, que parece dif�cil de hacer en principio.
    Para algunos caso especiales e importantes espacios, es suficiente verificar la convergencia d�bil para una clase restringida de funciones,
    definidas como lo siguiente.
\end{obs}

\begin{pro} Si $(\X,\A,\mu)$ es un medida finita y $1\leq p_1<p_2\leq\infty$ entonces
    \begin{equation}
        ||f||_{L^{p_1}}\leq c||f||_{L^{p2}} \qquad \text{Para cada } f\in L^{p^2}
    \end{equation}
    donde $c$ depende de $\mu(\X)$. Asi cada elemento de $L^{p_2}$ pertenece a $L^{p_1}$, y la convergencia fuerte en $L^{p_2}$
    implica convergencia fuerte en $L^{p_1}$
\end{pro}

\begin{proof} Sea $f\in L^{p_2}$ con $p_2<\infty$. Tomando $g=|f|^{p_1}$, se obtiene
\end{proof}

\begin{teo} Sea $(\X,\A,\mu)$ un espacio de medida y sea $\{f_n\}$ una sucesion de $f\in L^p(\X,\A,\mu)$
\end{teo}

