\section{Espacios \texorpdfstring{$L^p$}{Lp}}

\begin{dfn}[Espacios $L^p$, $L^p(\X,\A,\mu)$ o $L^p$]La familia de todas la posible funciones reales medible $f:\X\rightarrow\R$
    que satisface $$\int_\X{ | f(x)|^p\mu(dx)}<\infty$$
    Siendo $(\X,\A,\mu)$ un espacio de medida y p un n�mero real tal que $1\leq p<\infty$
\end{dfn}

\begin{obs} Si $p=1$ es habla del conjunto de todas las funciones integrables.
\end{obs}

\begin{dfn}[Norma de $f$, $|| f ||_{L^p}$]  Es el funcional
    \begin{equation}
        ||f||_{L^p}=\Big[ \int_\X{|f(x)|^p\mu(dx)}\Big]^{\frac{1}{p}}
    \end{equation}
\end{dfn}

\paragraph{Propiedades de la Norma}
    \begin{itemize}
        \item Dado $f\in L^p$
        \begin{equation}
            || f ||_{L ^p}=0\qquad \text{ si y solo si } f(x)=0\;\cs
        \end{equation}
        \item Sea $f\in L^p$ y $\alpha\in\R$
        \begin{equation}
            || \alpha f ||_{L^p}= | \alpha | \cdot ||_{L^p}f ||_{L^p}\qquad f\in L^{p}, \alpha\in\R
        \end{equation}
        \item Siendo $f,g\in L^p$
        \begin{equation}
            || f + g ||_{L^p}\leq ||  f ||_{L^p} +||g||_{L^p}\qquad f,g\in L^q
        \end{equation}
    \end{itemize}

De la propiedad L3 se deduce que $L^p$ es un espacio vectorial, ya que,
para $ f,g\in L^p$ y $\alpha\in\R$ entonces $(f+g)\in L^p$ y $\alpha f\in L^p$

\begin{dfn}[Distancia entre $f$ y $g$,$|| f + g ||_{L^p}$] Es el funcional
    $$|| f + g ||_{L^p}=\Big[\int_\X{|f(x)-g(x)|^p\mu(dx)}\Big]^{\frac{1}{p}}$$
\end{dfn}

\begin{obs} Es importante observar que el producto $fg$ de dos funciones $f,g\in L^{p}$ no es necesariamente un elemento de $L^p$
\end{obs}

\begin{dfn}[Espacio adjunto] Es $L^q(\X,\A,\mu)$ donde
    $$\Big(\frac{1}{p}\Big)+\Big(\frac{1}{q}\Big)=1$$
    Siendo $(\X,\A,\mu)$ un espacio de medida.
\end{dfn}

\begin{obs} El operador adjunto de $L_1$ es $L_\infty$, osea, las funciones medibles acotadas salvo en un conjunto de medida cero.
\end{obs}

\begin{dfn}[Producto escalar, $<f,g>$] De dos funciones de producto integrable se define como
    $$<f,g>=\int_\X{f(x)g(x)\mu(dx)}$$
siendo $f\in L^p$ y $g\in L^q$
\end{dfn}

\begin{dfn}[Inecuaci�n de  Cauchy-H�lder] Si $f\in L^{p}$ y $g\in L^{q}$ entonces
    $$|<f,g>|\leq ||f||_{L^p}  \cdot ||g||_{L^q}$$
\end{dfn}

\begin{obs}Para que la inecuaci�n tenga sentido cuando $f\in L^{1}$, $g\in L^{\infty}$, se toma para la norma
    $L^{\infty}$ la m�s peque�a constante c tal que
    $$|g(c) |\leq c$$
    Para casi todos $x\in\X$. Esta constante es llamada el supremo esencial de g. Tambi�n es una forma alternativa
    para definir las funciones de $L_{p\infty}$.
\end{obs}

\begin{obs} Se trabajar� la mayor�a de la veces en $L^1$; se indicar� en caso que la norma a usar sea contraria  a la de $L^1$.
    En otras palabras, $||f||=||f||_{L^1}$ . Observe que la desigualdad triangular en la norma $L^1$ es algunas veces una igualdad.
    Partiendo de la propiedad tenemos
    $$||f+g||=||f||+||g||\qquad f\geq 0, g\geq 0; f,g\in L^1$$
\end{obs}

Y para finalizar utilizaremos el concepto de $L^1$ para simplificar el teorema de Radon-Nikodym mostrado en el siguiente corolario 2

\begin{pro}Si $f_1$ y $f_2$ son funciones integrable tal que
$$\int_A{f_1(x)\mu(dx)}=\int_A{f_2(x)\mu(dx)}\qquad A\in\A$$
Entonces $f_1=f_2\;\cs$.
\end{pro}

\section{Convergencia de sucesiones de funciones}

\begin{dfn}[Convergencia Cesaro] La sucesi�n de funciones $\{f_n\}\rightarrow f$ si
    \begin{equation}
        \limi_{n\rightarrow\infty}{\frac{1}{n}\sum_{k=1}^{n}{<f_k,g>}}=<f,g> \qquad \forall g\in L^p
    \end{equation}
    donde $f_n, f\in L^p$ y $1\leq p<\infty$.
\end{dfn}

\begin{dfn}[Convergencia Debil]La sucesi�n de funciones $\{f_n\}\rightarrow f$ si
    \begin{equation}
        \limi_{n\rightarrow\infty}{<f_n,g>}=<f,g> \qquad \forall g\in L^p
    \end{equation}
    donde $f_n, f\in L^p$ y $1\leq p<\infty$.
\end{dfn}

\begin{dfn}[Convergencia Fuerte]La sucesi�n de funciones $\{f_n\}\rightarrow f$ si
    \begin{equation}
        \limi_{n\rightarrow\infty}{||f_k-f||_{L^p}}=0
    \end{equation}
    donde $f_n, f\in L^p$ y $1\leq p<\infty$.
\end{dfn}

\begin{obs} De la inecuaci�n de Cauchy-H�lder, se tiene
    $$|<f_n-f,g>|\leq ||f_n-f||_{L^p}\cdot||g||_{L^p}$$
    Y, asi, si $||f_n-f||_{L^p}$ converge a cero, entonces $<f_n-f,g>$ tambi�n. Entonces la convergencia fuerte implica convergencia d�bil.
    Usualmente la convergencia fuerte es m�s sencillo de probar. Sin embargo, la convergencia d�bil requiere
    la demonstraci�n que asegure que para toda $g\in L^p$, que parece dif�cil de hacer en principio.
    Para algunos caso especiales e importantes espacios, es suficiente verificar la convergencia d�bil para una clase restringida de funciones,
    definidas como lo siguiente.
\end{obs}

\begin{pro} Si $(\X,\A,\mu)$ es un medida finita y $1\leq p_1<p_2\leq\infty$ entonces
    \begin{equation}
        ||f||_{L^{p_1}}\leq c||f||_{L^{p2}} \qquad \text{Para cada } f\in L^{p^2}
    \end{equation}
    donde $c$ depende de $\mu(\X)$. Asi cada elemento de $L^{p_2}$ pertenece a $L^{p_1}$, y la convergencia fuerte en $L^{p_2}$
    implica convergencia fuerte en $L^{p_1}$
\end{pro}

\begin{proof} Sea $f\in L^{p_2}$ con $p_2<\infty$. Tomando $g=|f|^{p_1}$, se obtiene
\end{proof}

\begin{teo} Sea $(\X,\A,\mu)$ un espacio de medida y sea $\{f_n\}$ una sucesion de $f\in L^p(\X,\A,\mu)$
\end{teo}


